% CS305 Ethics - HW2
% Russell Miller

\documentclass{article}
\usepackage{anysize}
\usepackage{graphicx}

\marginsize{2cm}{2cm}{2cm}{2cm}

\title{CS305 HW2}
\author{Russell Miller}
\date{\today}

\begin{document}

\maketitle

\paragraph{1. Find a URL that describes the Hazelwood School District versus Kuhlmeier trial.\\}
http://cases.laws.com/hazelwood-v-kuhlmeier\\
This suit was brought to court by the Hazelwood School District because at East High School a student named Catherine Kuhlmeier attempted to print a story in the school newspaper about teen pregnancy.  She believed she had preserved anonymity. 
However, the principal - during the review process of the story - declared that the story was a violation of another student's privacy.
The newspaper editors felt that this was unconstitutional censorship.
The verdict was in favor of the school district because the first amendment doesn't apply to a public school the same way.\\
I would have to agree with the ruling, because protecting a student's privacy is more important than a news story in my opinion. It is good that the school district is able to do so.

\paragraph{2. Find a URL that describes what the Child Online Protection Act from 1998 is.\\}
http://kidshealth.org/parent/positive/family/net\_safety.html\\
It is briefly explained that the act is to protect kids on the internet. It does so by not allowing sites to obtain information about children without parental consent.

\paragraph{3. Find an article dated in 2007 that talks about what happened to the Child Online Protection Act after 1998 and what happened to it in 2007.\\} 
http://www.nytimes.com/2007/03/23/us/23porn.html\\
This is a news article that talks about how a federal judge voided the act. He found it to be ineffective and in violation of the first amendment.
Opinions are given about how this law would have crippled the internet, but also how children need better protection from pornography and predators.
Apparently this 1998 law never even went into effect because of a Supreme Court injunction in 2004.

\paragraph{4. Find a URL that describes the Communications Decency Act and its Supreme Court appearance.\\}
http://www.wired.com/politics/law/news/1997/06/4732\\
The CDA attempts to protect children by making it a crime to transmit ``indecent'' or ``patently offensive'' material, though these terms aren't defined in the act.
The court finds that the act ``tramples'' the first amendment and is far too broad.

\paragraph{5. Propose three tentative topics for your CS 305 presentation.\\}
\begin{itemize}
\item The hacker group Anonymous (and LulzSec)
\item The Trade Adjustment Act
\item Spy Software
\end{itemize}
\end{document}
