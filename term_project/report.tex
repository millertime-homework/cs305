% CS305 Term Paper

\documentclass{article}
\usepackage{anysize}
\usepackage{wasysym}
\usepackage{graphicx}

\marginsize{2cm}{2cm}{2cm}{2cm}

\title{Anonymous: Freedom Fighters or Cyber Punks?}
\author{Russell Miller}
\date{\today}

\begin{document}

\maketitle

\begin{center}
``We are Anonymous. We are Legion. We do not forgive. We do not forget. Expect 
us.''
\end{center}
%intro
\paragraph{}
Internet vigilantes, hackers, harassers.  How can we define what Anonymous 
\emph{is}?  We'll discuss what it means to be Anonymous.  It can be complicated.
We'll also take a look back at several events where people acted under the name 
Anonymous.  There have been many acts, with a wide spread of motives for them.
Lastly, we'll talk about becoming Anonymous and why I think, ethically speaking,
there is nothing wrong with doing so.

%what is
\paragraph{} 
It's important to realize that there are no official representatives of Anonymous.  
Anyone can be Anonymous.\footnote{
http://www.npr.org/2011/11/08/142124879/members-of-anonymous-share-set-of-values}
The act of declaring that you are Anonymous, however, removes your anonymity.
Because of this paradox, there are no rules to act as Anonymous.  This 
\emph{group}, which it arguably cannot be called, has no leaders.  This is similar 
to other facets of internet culture.  When you browse Youtube, hundreds -- or 
thousands -- of people argue and shout their opinions.  But what are the rules to 
prove credibility in your argument?  There are none.  Anyone can write obscenities 
in all caps in a Youtube video comment.  Granted, there are moderators there, and 
on public web forums.  Anonymous certainly takes it further.  

%protest
\paragraph{}
So what \emph{is} Anonymous?  What do they do?  Well, the short answer is that they
are hacktivists -- a portmanteau word combining hacker and activist.  By way of 
hacking software, they bring down web servers of large corporations and governments
in order to get attention.  This is typically delivered along with a video message 
explaining why the attack is happening.  If this doesn't seem right to you, think
about the following.  Activism as most people know it involves protesting with 
picket signs. To protest a corporation you go to the steps in front of their 
building an announce what it is that you don't agree with. You are not 
\emph{completely preventing} them from continuing to run their business, but you 
do something to get people's attention.  While hacking is not considered legal, 
shutting down a website for a few minutes or hours, or replacing a front page 
image with an Anonymous logo, does not completely prevent a website from operating.
It's merely a means to get attention.  The important thing is that the public
notices and hears the message behind why it was done.

%4chan
\paragraph{}
Let's really explore the history of Anonymous.  We'll start at the very 
beginning -- 4chan.  If you've never been to the website\footnote{
http://www.4chan.org}, I would advise you to approach with caution.  The subsection
of 4chan where Anonymous is said to have originated\footnote{
http://www.theregister.co.uk/2011/07/07/anonymous\_feature/} is ``/b/'', which is 
just the ``Random'' posts.\footnote{http://boards.4chan.org/b/}
I personally do not frequent these message boards.  In researching for this report,
I perused for a while and some of the content is completely ridiculous, and in many
cases not appropriate to view at work, school, or your mother's house.  But anyone
can post images and comments.  If you don't want to register a name, your
posts are published with the name ``Anonymous'' with a capital A.  This is
the beginning of how Anonymous acts were organized.

%turner/forcand
\paragraph{}
And, with that, we have events with the Anonymous label.  The first I was able to
find took place in 2006, when a white supremacist radio host Hal Turner's website
was taken down.\footnote{
http://dockets.justia.com/docket/new-jersey/njdce/2:2007cv00306/198438/}
He attempted to sue the message boards, but the trial was dropped
when Turner didn't respond to a letter from the court.  Unfortunately, I have
no idea what the motivation was here.  In some ways, this attack contradicts the
general ideal in favor of free speech that Anonymous is known for.  However, that
is a perfect example of how there are no rules about who is to be attacked or why.
The next event was in 2007 and ended with a child predator named Chris Forcand
being arrested.\footnote{
http://cnews.canoe.ca/CNEWS/Crime/2007/12/07/4712680-sun.html}
Anonymous helped to track down Forcand by assuming the guise of a
young girl having inappropriate internet chats with him.  I was shocked when I
first read about this.  I had never heard the story, and was not aware there were
such meaningful acts that took place.  At this point, it seems that Anonymous
really reaches out for a good cause.

%chanology
\paragraph{}
Next we continue to probably Anonymous's biggest movement, which is that against
the Church of Scientology. It's been dubbed Project Chanology and began in 2008
with attacks on their websites and offices.  It was originally spurred by the
church's reaction to the leak of an interview with Tom Cruise about Scientology
on Youtube.\footnote{
http://www.guardian.co.uk/technology/2008/feb/04/news}
The church wanted it taken down, and Anonymous felt that this was the church
censoring itself.  This started what could be considered an ongoing war between
Anonymous and the Church of Scientology, as there are many practices Anonymous
disagrees with.  Anonymous also moved into other facets of peaceful protest.
They wish to ``Save people from Scientology by reversing the 
brainwashing.''\footnote{
http://www.ibls.com/internet\_law\_news\_portal\_view.aspx?s=latestnews\&id=1972}

%epilepsy
\paragraph{}
By this point, you may be telling yourself that while Anonymous seems to be
controversial in the methods of driving a point home (i.e. hacking into websites),
they are doing it for a cause.  Sadly, that view is about to shift. With a 
leaderless ubiquitous group like Anonymous there are bound to be outliers. Since
anyone can hold their own Anonymous attack or protest, it's possible for attacks
to happen without good cause.  This brings us to the attack on the Epilepsy
Foundation forum website.  Flashing images and other content were posted on the
website in order to trigger reactions from its users.\footnote{
http://www.wired.com/politics/security/news/2008/03/epilepsy}
It cannot be said for certain that Anonymous carried this out, as some people
claimed it was the Church of Scientology attempting to deface Anonymous.\footnote{
http://www.news.com.au/technology/anonymous-attack-targets-epilepsy-sufferers/story-e6frfro0-1111115935811}
If it was the Church of Scientology, this sort of attack would fall under its
``fair game'' policy, which states that enemies of the church ``can be punished 
and harassed using any and all means possible.''\footnote{
http://en.wikipedia.org/wiki/Fair\_Game\_(Scientology)\\
Note: It seems most Scientology info is published on paper, not the internet.
Thus my source is Wikipedia.}

%sohh
\paragraph{}
Next we'll investigate the defacement of the website Support Online Hip-Hop
(SOHH), another attack without merit.  This time rather than deny it, 
self-proclaimed members of Anonymous claimed to be responsible.  Racist images
and slurs were scripted into the website's main page, and DDoS attacks brought
down the server.\footnote{
http://www.mtv.com/news/articles/1590117/hiphop-sites-hacked-by-apparent-hate-group.jhtml}
If you've been rooting for Anonymous up to this point, this is very
disappointing.  But in the years that follow this attack, Anonymous became
much more mature.

%iran
\paragraph{}
The next act Anonymous took part in was the 2009-2010 Iranian election protests.
There was controversy surrounding the election, including accusations of fraud.
Protests started and violence was rampant.  Much of the protesting was organized
online via Twitter or other means, and the Iranian government attempted to censor
it.  Thus Anonymous Iran was born, a website to help support the protesters.
This site was also backed by a torrent website called The Pirate Bay.\footnote{
http://news.ninemsn.com.au/technology/827036/internet-underground-takes-on-iran}

%au
\paragraph{}
Also in 2009, Australia continued to face internet censorship problems, with
the government attempting to set up a censorship filter.\footnote{
http://www.abc.net.au/news/2009-02-27/senate-poses-tough-hurdle-for-internet-filtering/1603944}
Anonymous responded by hacking in to the Australian Government Classification
website, and redirecting users to a message mocking censorship.\footnote{
http://www.atomicmpc.com.au/News/140964,breaking-news-australian-government-website-hacked.aspx}
This lead to Operation Didgeridie and additional attacks, including a DDoS on
the prime minister's website.\\
Anonymous came back to Australia only a few months later with Operation Titstorm,
a reaction to censorship of certain pornographic content. A series of DDoS 
attacks on government websites, including the Australian parliament, as well
as e-mail spam attacks, were carried out. 

%payback
\paragraph{}
This next \emph{operation} is somewhat comical, at least in its roots. There
are websites providing pirated content all over the world.  In 2010, after
many of these websites received takedown notices and ignored them, Aiplex
Software was hired to -- believe it or not -- hack the sites.  They attempted
to use DDoS attacks to bring down these copyright-infringing suspects.\footnote{
http://www.theregister.co.uk/2010/09/20/4chan\_ddos\_mpaa\_riaa/}
Anonymous caught onto this and launched Operation Payback.  They wanted to
take down Aiplex, but just hours before their planned attack would take place,
someone else launched a DDoS assault.  So Anonymous decided to go for the MPAA
and RIAA instead, as well as other copyright organizations and lawyers. 

%assange and brad
\paragraph{}
An enormous cause for censorship dispute formed back in 2006, and it is called 
Wikileaks.  Lead by Julian Assange, Wikileaks has been publishing classified 
information in order to get the truth out. Well the funding for the site, from 
donations, got cut off by several companies including MasterCard, Visa, and
PayPal due to pressure felt because of US documents being released.  This caused
Anonymous to retaliate with Operation Avenge Assange, hitting PayPal with a DDoS 
and bringing down the MasterCard and Visa websites.\footnote{
http://www.businessinsider.com/cyber-hackers-that-took-down-swiss-bank-site-have-now-taken-down-mastercardcom-2010-12}\\
Bradley Manning was a US soldier suspected of providing classified US documents
to WikiLeaks.  He was arrested and brought to a detention center.  There his
treatment was what many consider to be unlawful.  Anonymous responded to this
with dox attacks on the Chief Warrant Officer of the detention center and on
the Defense Department Press Secretary, calling it Operation Bradical.\footnote{
http://www.thenewnewinternet.com/2011/03/07/anonymous-launches-operation-bradical/}
A dox attack is very personal, including release of personal information, e-mail 
spam, ordering hundreds of pizzas or magazine subscriptions.  

%wrap-up
\paragraph{My Opinion\\}
This is a laundry list of events that have taken place.  There are a myriad of
other events that I would love to continue talking about, but the key point of
controversy related to the style of attack that Anonymous executes is the fact
that hacking into a website is illegal.  Distributed Denial of Service attacks
are illegal.  Is this civil disobedience?  I stated earlier that I feel a DDoS
attack is an acceptable way to get the attention of the public.  High traffic
websites could potentially lose some of their revenue for an outage caused by
one of these attacks, but in reality the server is not destroyed and it will be
online again eventually.  A message had to be delivered somehow.  I personally
feel like more people need to help Anonymous's cause and speak out against
injustice and censorship.  There will be less noble actions carried out by self-
proclaimed Anonymous members, and I will disagree with them.  But that doesn't
change the fact that Anonymous has helped this country and others in a big way.

%join!
\paragraph{}
One facet of communication that Anonymous now uses is AnonOps.\footnote{
http://anonops.blogspot.com/}
They speak on Internet Relay Chat (IRC) and sometimes relay messages on 
Facebook, Twitter, and Youtube.  Peaceful protest with a picket sign is always a 
great contribution but not something I would be personally willing to take part 
in. There are means that I will not discuss for hopefully keeping yourself out 
of jail while supporting the Anonymous operations.  Recently our government
has been trying to pass laws to censor the internet here in the US, and if it
came down to it an Anonymous attack would be more than appropriate.  

\paragraph{}
In closing, I will say that I am thankful to live in a country where we still
have a lot of freedom and plenty of opportunities.  In no way do I suffer during
my day-to-day life.  There are definitely a lot of countries that cannot say 
that.  But if our government continues to suppress and they manage to take more
of our freedom away, and we begin to lose the power to resist it, I will take a
stand.   If you see nothing, if the crimes of this government remain unknown to 
you then I would suggest you carry on with your life.  But if you agree that the
government should not overtake its people, then together we shall call ourselves
Anonymous and rise up for freedom!

\end{document}
